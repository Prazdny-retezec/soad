% +-------------------+
% | Rozložení stránek |
% +-------------------+
\usepackage[
    a4paper,
    width=150mm,
%  height=220mm,
    top=25mm,
    bottom=25mm
]{geometry}


% +------+
% | Font |
% +------+
\usepackage{fontspec}
\setmainfont{Linux Libertine O}


% +---------+
% | Čeština |
% +---------+
\usepackage[czech]{babel}


% +------------------------------------+
% | programovací jazyk Lua - LuaLaTeX  |
% +------------------------------------+
\usepackage{luacode}
\newcommand{\luavar}[1]{\directlua{tex.sprint(#1)}}
\begin{luacode*}
-- vrací uživatelem zadané datum nebo aktuální datum
function get_date()
  if (date == nill or date == '') then
    return tex.print('\\today')
  else
    return tex.print(date)
  end
end

-- vrací všechny autory dokumentu( jméno + obor) naformátované pro TeX
function get_authors()
  local result = ""
  for i, author in ipairs(authors) do
    result = result .. author.name .. "&&&&&" .. author.program .. "\\\\"
  end
  return result
end

-- vrací počet autorů
function get_authors_count()
    local result = 0
    for i, author in ipairs(authors) do
        result = result + 1
    end
    return result
end

-- vrací jména všech autorů oddělená čárkami
function get_author_names()
    local result = ""
    if get_authors_count() == 1 then
        return authors[1].name
    else
        for i, author in ipairs(authors) do
            result = result .. author.name .. ", "
        end
        result = result:sub(1, -3)
        return result
    end
end

-- podmíněné výpisy
function printBiblio()
  local output = "\\newpage \\section{Literatura} \\printbibliography[heading=none]"
  if (PRINT_BIBLIO) then
    return tex.print(output)
  end
end

function printTablesList()
    if (PRINT_TABLES) then
        return tex.print("\\listoftables")
    end
end

function printFiguresList()
    if (PRINT_FIGURES) then
        return tex.print("\\listoffigures")
    end
end

function printSnippetsList()
    if (PRINT_SNIPPETS) then
        return tex.print("\\lstlistoflistings")
    end
end
\end{luacode*}


% +---------+
% | seznamy |
% +---------+
\usepackage{multicol} % vícesloupcový seznam


% +---------+
% | chemie  |
% +---------+
\usepackage{chemformula}


% +---------+
% | obrázky |
% +---------+
\usepackage{graphicx}
\graphicspath{{./images/}}
\usepackage{svg}
\usepackage{subcaption}
\usepackage{wrapfig}


% +-------+
% | barvy |
% +-------+
\usepackage{color}
\definecolor{dkgreen}{rgb}{0,0.6,0}


% +--------+
% | odkazy |
% +--------+
\usepackage{hyperref}
\hypersetup{
    colorlinks=true,
    citecolor=black,
    filecolor=black,
    linkcolor=black,
    pdftitle={\luavar{title}},
    urlcolor=teal
}

% +------------------------------+
% |         bibliografie         |
% +------------------------------+
% | používá se norma ČSN ISO 690 |
% +------------------------------+
\usepackage{csquotes}
\usepackage[style=iso-authoryear,
    hyperref=true,
    url=false,
    isbn=false,
    backref=true]{biblatex}
\addbibresource{biblio.bib}
% převedení bibliografie do kapitálek
%\renewcommand{\mkbibcompletename}[1]{\textsc{#1}}  % afektuje kromě bibliografie i citace
\renewcommand*{\mkbibnamegiven}[1]{\textsc{#1}}


% +-----------------------+
% | sazba zdrojových kódů |
% +-----------------------+
\usepackage{listings}
\renewcommand{\lstlistingname}{Zdrojový kód}
\renewcommand{\lstlistlistingname}{Seznam zdrojových kódů}

% čeština ve zdrojovém kódu
\lstset{extendedchars}
\begingroup
\catcode0=12 %
\makeatletter
\g@addto@macro\lst@DefEC{%
    \lst@CCECUse\lst@ProcessLetter
    ěščřžĚŠČŘŽťŤďĎňŇů
    ^^00%
}%
\endgroup

% stylování zdrojového kódu (barvy klíčových slov, ...)
\lstdefinestyle{mystyle}{
    backgroundcolor=\color{white},
    commentstyle=\color{dkgreen},
    keywordstyle=\color{orange},
    numberstyle=\tiny\color{gray},
    stringstyle=\color{purple},
    basicstyle=\ttfamily\footnotesize,
    breakatwhitespace=false,
    breaklines=true,
    captionpos=b,
    keepspaces=true,
    numbers=left,
    numbersep=5pt,
    showspaces=false,
    showstringspaces=false,
    showtabs=false,
    tabsize=2
}
\lstset{style=mystyle}


% +--------------------------+
% | záhlaví a zápatí stránek |
% +--------------------------+
\usepackage{fancyhdr}


% +----------------------------+
% | vlastní použitelné příkazy |
% +----------------------------+
\newcommand\lorem{
    Lorem ipsum dolor sit amet, consectetur adipiscing elit, sed do eiusmod tempor incididunt ut labore et dolore magna aliqua. Ut enim ad minim veniam, quis nostrum exercitationem ullam corporis suscipit laboriosam.
}
\newcommand\mistrjan{
    Krajani a druzi moji, ach, smutno mi je, trudno, neveselo, chmurami jsem zavalen, hanbou jsem zdrcen a jest mi za co pykat. Maje na mysli jen dobro, bez rozmyslu jsem vnesl do jazyka velkou lotrovinu a tak zavinil historickou nehodu. Dlouho se to tutlalo, teprve tato epocha celou moji vinu naplno vyjevila. Uznal jsem to a kaji se. V troufalosti ducha, a usiluje pouze o to, aby bylo lze rychleji rozmluvy, ano i knihy, skripta a lejstra zapisovati, jsem vymyslel potrhlou a krutou fintu. Brkem z husy jsem litery a, c, d, e, i, n, o, r, s, r, u, y, z pobodal a zle poranil.
}


% +-------------------+
% | Titulní stránka   |
% +-------------------+
\newcommand\mytitlepage{
    \begin{titlepage}
    \begin{center}
    \Large{
        \luavar{university}\\
        \luavar{faculty}\\
    }
    \vspace{0.2cm}
    \hrule

    \vfill
    \Huge{
        \textbf{\luavar{title}}
    }

    \vspace{1.5cm}

    \LARGE {
        \textbf{\luavar{document_type}}\\
    }

    \vspace{0.2cm}

    \Large {
        \luavar{subject}\\
    }

    \vfill

    \vspace{0.8cm}

    \Large{
        \begin{tabular}{lccccl}
        \directlua{tex.print(get_authors())}
        \end{tabular}
    }

    \vspace{1.5cm}

    \Large{
        \luavar{place}, \directlua{get_date()}
    }
    \end{center}
    \end{titlepage}
}

% +------------------------------------+
% | Nové prostředí pro vložení práce   |
% +------------------------------------+
\newenvironment{teamwork}
{% begin

% PDF meta data
    \hypersetup{
        pdfinfo={
            Title={\luavar{title}},
            Author={\luavar{get_author_names()}},
            Subject={\luavar{subject}},
            Keywords={\luavar{keywords}}
        }
    }

    % titulní stránka
    \mytitlepage

    % obsah
    \tableofcontents
    \thispagestyle{empty}
    \newpage
    \setcounter{page}{1}

    % záhlaví a zápatí
    \pagestyle{fancy}
    \fancyhead{}
    \fancyfoot{}
    \fancyhead{\nouppercase{\leftmark}\hfill\thepage}
    \setlength{\headheight}{14.5pt}
    }
    {% end

    % bibliografie
    \directlua{printBiblio()}

    % seznam obrázků
    \directlua{printFiguresList()}

    % seznam tabulek
    \directlua{printTablesList()}

    % seznam zdrojových kódů
    \directlua{printSnippetsList()}
}
